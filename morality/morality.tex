\documentclass[11pt, oneside]{article}   	% use "amsart" instead of "article" for AMSLaTeX format
\usepackage{geometry}                		% See geometry.pdf to learn the layout options. There are lots.
\geometry{letterpaper}                   		% ... or a4paper or a5paper or ... 
\usepackage{graphicx}				% Use pdf, png, jpg, or eps§ with pdflatex; use eps in DVI mode
\usepackage{amssymb}

\title{Can morality by taught to an AI?}
\author{Abhi Agarwal}
\date{}							% Activate to display a given date or no date

\begin{document}
\maketitle

\par \textbf{How do we define morality?}
\begin{itemize}
	\item Distinction between right and wrong. It is the determination of what should be done and what should not be done.\footnote{http://carm.org/dictionary-morality}
	\item Biblically, morals are derived from God's character and revealed to us through the Scriptures.\footnote{http://carm.org/dictionary-morality}
	\item Law or a legal system is distinguished from morality or a moral system by having explicit written rules, penalties, and officials who interpret the laws and apply the penalties.\footnote{http://plato.stanford.edu/entries/morality-definition/}
	\item The term comes from both Latin and Greek. In Latin it is ``mores", and in Greek it is ``ethos". Each derives their meaning from the idea of custom.\footnote{http://www.slideshare.net/dborcoman/chapter1-9042561} Therefore morality can refer to:
	\begin{itemize}
		\item Customs
		\item Precepts
		\item Practices of people and cultures
		\item Virtues, values, and principles of people
	\end{itemize}
	\item The concept of being moral seeks to establish principles of right behavior. It can be used to serve as a guide for individuals and groups. 
\end{itemize}

\par \textbf{What is it to be a moral person?}

\par \textbf{What is the nature of morality?}

\par \textbf{Why do we need morality?}

\par \textbf{What function does morality play?}

\par \textbf{How do I know what is good?}

\par \textbf{What do morals depend upon?}
\begin{itemize}
	\item Morals differ among cultures, and there are morals that are relative, i.e., dependent upon situations and context.\footnote{http://carm.org/dictionary-morality}
\end{itemize}

\par \textbf{What can the term morality be used for?}
\begin{itemize}
	\item Descriptively to refer to some codes of conduct put forward by a society (some other group, such as a religion or accepted by an individual for her own behavior)\footnote{http://plato.stanford.edu/entries/morality-definition/}
	\item Normatively to refer to a code of conduct that, given specified conditions, would be put forward by all rational persons.\footnote{http://plato.stanford.edu/entries/morality-definition/}
\end{itemize}

\par \textbf{What are the moral characteristics?}
\begin{itemize}
	\item Being Honest, Truthful, Trustworthy
	\item Having Integrity
	\item Being Caring/Compassionate/Benevolent
	\item Doing One's Civic Duty
	\item Having Courage
	\item Being Willing to Sacrifice
	\item Maintaining Self-Control
	\item Being just and fair
	\item Being Cooperative
	\item Being Persevering/ Diligent
	\item Keeping Promises
	\item Doing no harm
	\item Pursuing excellence/takes pride in work
	\item Taking personal responsibility
	\item Having Empathy
	\item Benefiting others 
	\item Having Respect for others
	\item Having Patience
	\item Being Forgiving
	\item Making Peace
	\item Having Fidelity/Loyal
	\item Respecting Autonomy
	\item Being Tolerant
	\item Having Self-respect
	\item Competitiveness
	\item Valuing Life
\end{itemize}

\par \textbf{The three laws of Robotics}
\begin{itemize}
	\item A robot may not injure a human being or, through inaction, allow a human being to come to harm.
	\item A robot must obey the orders given to it by human beings, except where such orders would conflict with the First Law.
	\item A robot must protect its own existence as long as such protection does not conflict with the First or Second Law.\footnote{http://en.wikipedia.org/wiki/Three\_Laws\_of\_Robotics}
\end{itemize}

\par \textbf{Observations}
\begin{itemize}
	\item When the set of morals we follow are written down do they become rules? Is law different to morality because it is written down?
	\item Do we need a sense of morality to make a judgement?
	\item Are moral principles absolute?
	\item How do we investigate which values and virtues are important for a worthwhile life in society?
\end{itemize}

\end{document}  